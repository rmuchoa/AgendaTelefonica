\documentclass{acm_proc_article-sp}
\usepackage[utf8]{inputenc}
\usepackage{url}
\graphicspath{{./imagens/}}
\begin{document}

\title{Utilizando SOAP para construir Web Service
\titlenote{(Does NOT produce the permission block, copyright information nor page numbering). For use with ACM\_PROC\_ARTICLE-SP.CLS. Supported by ACM.}}
\subtitle{[Extended Abstract]
\titlenote{A full version of this paper is available as
\textit{Author's Guide to Preparing ACM SIG Proceedings Using
\LaTeX$2_\epsilon$\ and BibTeX} at
\texttt{www.acm.org/eaddress.htm}}}

\numberofauthors{5} 

\author{
	\alignauthor
	Juliano Rodovalho\titlenote{Juliano R. Macedo}\\
		   \affaddr{Fedaral University of Pampa}\\
		   \affaddr{Av Tiarajú, 810}\\
		   \affaddr{Alegrete - RS}\\
		   \email{j.rodovalho.m@gmail.com}
	% 2nd. author
	\alignauthor
	Lucas P. Capanelli\titlenote{Lucas P. Capanelli}\\
		   \affaddr{Fedaral University of Pampa}\\
		   \affaddr{Av Tiarajú, 810}\\
		   \affaddr{Alegrete - RS}\\
		   \email{webmaster@marysville-ohio.com}
	% 3rd. author
	\alignauthor  Rafael T. Amorim\\
		   \affaddr{Fedaral University of Pampa}\\
		   \affaddr{Av Tiarajú, 810}\\
		   \affaddr{Alegrete - RS}\\
		   \email{jrtadf@gmail.com}
	\and  % use '\and' if you need 'another row' of author names
	% 4th. author
	\alignauthor Renan M. Uchoa\\
		   \affaddr{Fedaral University of Pampa}\\
		   \affaddr{Av Tiarajú, 810}\\
		   \affaddr{Alegrete - RS}\\
		   \email{lleipuner@researchlabs.org}
	% 5th. author
	\alignauthor Helison R. Teixeira\\
		   \affaddr{Fedaral University of Pampa}\\
		   \affaddr{Av Tiarajú, 810}\\
		   \affaddr{Alegrete - RS}\\
		   \email{helisonreus@gmail.com}
%	\alignauthor 
}


\maketitle
\begin{abstract}
		This article is intended to show an overview about webservices tecnologies.	
\end{abstract}


% A category with the (minimum) three required fields
\category{H.4}{Information Systems Applications}{Miscellaneous}
%A category including the fourth, optional field follows...
\category{D.2.8}{Software Engineering}{Metrics}[complexity measures, performance measures]

\terms{Theory}

\keywords{Webservices, soap, xml} 

\section{Introdução}
		Web Services trouxeram uma nova forma de construir Sistema Distribuídos, e com a Service Oriented Architeture (SOA), uma nova filosofia de integrar sistemas e serviços. Mas para compreender como os Web Services se comportam, é preciso dominar, antes de tudo, um pouco sobre o Service Oriented Architeture Protocol (SOAP) que ele implementa, a tecnologia Extensible Markup Language (XML) que é utilizada pelo protocolo para trafegar as mensagens entre os processos, e o próprio Hypertext Transfer Protocol (HTTP) que é utilizado para encapsular o protocolo SOAP de comunicação durante as requisições cliente-servidor.
		
		
\section{XML}
		XML é uma linguagem de marcação extensível, projetada para transportar, armazenar e processar dados. O XML representa um conjunto de especificações especificadas e relatadas pelo W3C e outros. O XML possui um ancestral em comum com o Standard Generalized Markup Language (SGML). Uma das caracteristicas do GSGML era a separação do formato do conteudo. Se um documento foi produzido para o formato A4 ou carta, por exemplo, o formato era descrito independentemente do conteudo do documento. O mesmo documento portanto pode ser enviado em vários formatos sem mudar seu conteúdo. O pricipio das linguagens de marcação são aplicadas para os Web Services através da separação da instancia do documento, que contém os dados, e o esquema, que descreve a estrutura dos dados e os tipos, incluindo informações de semântica muito úteis para fazer o mapeamento do documento para várias linguagens de programação e sistemas de software. 
		
		O XML é similar ao HTML, contém elementos, atributos e valores. Bem feitos os documentos em XML podem ser exibidos em navegadores web, porém, esse aspecto do XML não é muito relevante para os Web services. O HTML contém um conjunto finito de elementos e atributos, porém o XML permite ser definido tantos conjuntos quanto necessários.
		
		Os elementos e atributos de um arquivo XML definem independentemente seus tipos e estruturas de informação para o tipo de dado que eles contém, incluindo a capacidade de modelar dados e estruturas especificas à um dominio de software. (Um dominio de software pode ser uma linguagem de programação, um middleware, um pacote de aplicações, ou um sistema de gerenciamento de dados). A transformação de uma representação genérica de dados contida em um XML em um aplicativo, ou um dominio de software, a representação específica de dados é o aspecto principal dos Web serices.
		
		Os Web Services utilizam esquemas em XML para a validação de mensagens. O SOAP do computador que esta enviando a mensagem transforma seus dados da forma nativa para o esquema XML pré-definido contido no arquivo WDSL para texto, pontos flutuantes, e outros, usando o mapeamento das tabelas. O Mapeamento das tabelas associa tipos de dados nativos com os tipos de dados correspondentes no XML. Padrões de mapeamento estão disponíveis para Java, Visual Basic, CORBA, e outros tipos mais comuns de tipos de sistema. Muitas ferramentas XML estao disponíveis para definir mapeamentos customizados ou especiais. O computador que recebe o processo SOAP executa a transformação reversa, mapeando os tipos de dados do XML para os seus tipos de dados correspondentes.\cite{UNDERWEBSERVICES}
		
		A sintaxe usada nas tecnologias de Web services especifica como os dados são genericamente representados, define como e com quais qualidades de serviço os dados são transmitidos, e detalha como o serviço é enviado e recebido. As implementações de Web service decodificam esses vários bits de XML para interagir com várias aplicações e dominios de software por baixo do serviço.
		
\section{HTTP}
		O Hypertext Transfer Protocol(HTTP) é um protocolo de nível de aplicação, utilizado em sistemas distribuidos, colaborativos e sistemas de informação multimídia. É um protocolo genérico, e não proprietario, pode ser usado para muitos objetivos, além do hipertexto, como domínios de servidores e gerenciamento de objetos de sistemas distribuidos, através da extensão de seus métodos de requisição, códigos de erro e cabeçalhos. Uma característica do HTTP é a escrita e negociação da representação de dados, habilitando os sistemas a fazer a construção dos dados a serem transferidos independentemente.\cite{HTTP-1.1}

\section{Webservices}
		
		Web Services são formas de utilizar XML para mapear programas, objetos ou banco de dados ou para funções de negócio abrangentes. Usando um documento XML criado na forma de uma mensagem, um programa envia uma requisição para um Web Service através da rede, e eventualmente recebe uma resposta também na forma de um documento XML. Os padrões Web Service definem o formato das mensagens, especificam a interface para qual as mensagens são enviadas, descrevem as convenções para mapear os conteúdos das mensagens dentro e fora dos programas que implementam o serviço, e definem mecanismos para publicar e para descobrir interfaces de Web Services.
		
		Web Services oferecem uma camada de abstração acima dos sistemas existentes, assim como Servidores de Aplicação, CORBA, servidores .Net, aplicativos mensageiros e pacotes de aplicação. Web Services trabalham sobre 
		uma camada de abstração similar a da Internet e são compativeis de realizar a transição entre qualquer sistema operacional, plataforma de hardware ou linguagem de programação, assim como a WEB é.
		
		Ao contrário dos sistemas distribuidos existentes, os Web Services são adaptados para a Web. A maioria das tecnologias de computação distribuida incluem os protocolos de comunicação em seu escopo. Com Web Services os protocolos de comunicação já estão prontos e disponíveis por toda a Web. \cite{UNDERWEBSERVICES}

	\subsection{O que são?}
		
		
		
		
	
%Helison -> Arquitetura
	\subsection{Arquitetura}
		
	\subsection{Funcionamento}
	


		
\section{SOAP}
	\subsection{O que é?}
	
		Simple Object Access Protocol ou procotolo simples de acesso a objetos, conhecido como SOAP, é um protocolo de webservices que utiliza o formato XML para troca de mensagens. O padrão do protocolo SOAP é reconhecido pela W3C desde 2003 em sua versão 1.2. Inicialmente tendo a empresa de tecnologia Microsoft como seu maior apoiador, no entanto é governado pela W3C atualmente.
		
		O SOAP é orientado a métodos, estes métodos podem ter ou não parametros, portanto é necessário o conhecimento da assinatura dos métodos. A execução dos métodos é realizada no servidor.
		
		Existem duas versão do SOAP: 1.1 e 1.2. Dentre as diferenças, pode-se mencionar que a versão 1.1 suporta apenas o HTTP com transporte de mensagens, enquanto a versão 1.2 permite a utilização de outros protocolos de transporte além do HTTP. \cite{WEBSERVICESZEND}
		
	\subsection{Funcionamento}
	
		Para descrever um webservice é utilizado vocabulário em XML chamado Web Service Services Description Language conhecido como WSDL, que define como o webservice é acessado, quais as operações existem, como as mensagens são transferidas e a estrutura das mensagens. O WDSL não é obrigatório para trabalhar com webservices, ele faz uma parte integral do perfil do básico WS-I da organização \emph{OASIS Web Services Interoperability} \footnote{A Web Services Interoperability Organization (WS-I) é uma organização aberta com objetivo de estabelecer as melhores práticas para a interoperabilidade de web services, para grupos selecionados de padrões de web services, através de plataformas, sistemas operacionais e linguagens de programação.\cite{OASIS-WS-I-SITE}} e facilita o trabalho. 
\section{Implementação}
%Falar sobre a implementação do sistema, no caso falar como o Web services foi criado e afins 

\section{Conclusão}
This paragraph will end the body of this sample document.
Remember that you might still have Acknowledgments or
Appendices; brief samples of these
follow.  There is still the Bibliography to deal with; and
we will make a disclaimer about that here: with the exception
of the reference to the \LaTeX\ book, the citations in
this paper are to articles which have nothing to
do with the present subject and are used as
examples only.
%\end{document}  % This is where a 'short' article might terminate

%ACKNOWLEDGMENTS are optional
\section{Acknowledgments}
This section is optional; it is a location for you
to acknowledge grants, funding, editing assistance and
what have you.  In the present case, for example, the
authors would like to thank Gerald Murray of ACM for
his help in codifying this \textit{Author's Guide}
and the \textbf{.cls} and \textbf{.tex} files that it describes.

%
% The following two commands are all you need in the
% initial runs of your .tex file to
% produce the bibliography for the citations in your paper.
\bibliographystyle{abbrv}
\bibliography{sigproc}  % sigproc.bib is the name of the Bibliography in this case
% You must have a proper ".bib" file
%  and remember to run:
% latex bibtex latex latex
% to resolve all references
%
% ACM needs 'a single self-contained file'!
%
%APPENDICES are optional
%\balancecolumns
\appendix
%Appendix A
\section{Headings in Appendices}
The rules about hierarchical headings discussed above for
the body of the article are different in the appendices.
In the \textbf{appendix} environment, the command
\textbf{section} is used to
indicate the start of each Appendix, with alphabetic order
designation (i.e. the first is A, the second B, etc.) and
a title (if you include one).  So, if you need
hierarchical structure
\textit{within} an Appendix, start with \textbf{subsection} as the
highest level. Here is an outline of the body of this
document in Appendix-appropriate form:
\subsection{Introduction}
\subsection{The Body of the Paper}
\subsubsection{Type Changes and  Special Characters}
\subsubsection{Math Equations}
\paragraph{Inline (In-text) Equations}
\paragraph{Display Equations}
\subsubsection{Citations}
\subsubsection{Tables}
\subsubsection{Figures}
\subsubsection{Theorem-like Constructs}
\subsubsection*{A Caveat for the \TeX\ Expert}
\subsection{Conclusions}
\subsection{Acknowledgments}
\subsection{Additional Authors}
This section is inserted by \LaTeX; you do not insert it.
You just add the names and information in the
\texttt{{\char'134}additionalauthors} command at the start
of the document.
\subsection{References}
Generated by bibtex from your ~.bib file.  Run latex,
then bibtex, then latex twice (to resolve references)
to create the ~.bbl file.  Insert that ~.bbl file into
the .tex source file and comment out
the command \texttt{{\char'134}thebibliography}.
% This next section command marks the start of
% Appendix B, and does not continue the present hierarchy
\section{More Help for the Hardy}
The acm\_proc\_article-sp document class file itself is chock-full of succinct
and helpful comments.  If you consider yourself a moderately
experienced to expert user of \LaTeX, you may find reading
it useful but please remember not to change it.
\balancecolumns
% That's all folks!
\end{document}
